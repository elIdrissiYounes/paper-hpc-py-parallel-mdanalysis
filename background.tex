\label{background}
\obnote{The background still needs: other parallel analysis libraries, work on accessing files in parallel; someone must have written something...}
Long Tail phenomena, whereby a small proportion of task stragglers significantly impede job completion time is a big challenge toward achieving improved performance \cite{Garraghan2016}.
Long Tail phenomena is a known problem in other frameworks such as Spark \cite{Kyong2017,Ousterhout2017,Gittens2016}, Google\textsc{\char13}s MapReduce \cite{Dean2004}, Hadoop \cite{Dean2004}, and cloud data centers \cite{Schmidt2016}.

Straggler root-cause has both internal and external factors. 
Internal factors include heterogeneous capacity of worker nodes, and resource competition due to other tasks running on the same worker node. 
External factors include resource competition due to co-hosted applications, input data skew, remote input or output source being too slow and faulty hardware \cite{Chen2014}.

The straggler node due to the faulty hardware or mis-configuration can lead to stragglers \cite{Dean2004}. 
Garbage collections~\cite{Kyong2017,Ousterhout2017}, JVM positioning to cores~\cite{Kyong2017}, the delay's introduced while the task moves from the scheduler to executing~\cite{Gittens2016}, Disk I/O during shuffling, Java's just-in-time compilation~\cite{Ousterhout2017}, and output skew \cite{Ousterhout2017}. 
In addition to these reasons, stragglers on Spark have been attributed to the overall performance of the worker or competition between resources \cite{Yang2016}.
Garraghan et al. \cite{Garraghan2016} reported high CPU utilization, disk utilization, unhandled I/O access requests and network package loss as the most frequent reasons for stragglers on Virtualized Cloud Data-centers.

Tuning resource allocation and tuning parallelism such as breaking the workload into many small tasks that are dynamically scheduled at runtime \cite{Rosen2012}, slow Node-Threshold \cite{Dean2004}, speculative execution \cite{Dean2004}, sampling or data distribution estimation techniques, SkewTune to avoid data imbalance \cite{Kwon2012}, dynamic work rebalancing \cite{Schmidt2016}, blocked time analysis \cite{Ousterhout2015}, and Intelligent scheduling \cite{AWE-WQ2014} are among a wide variety of approaches that are trying to detect and mitigate stragglers. 

However, a comprehensive solution will involve tricks to avoid stragglers. 
This work focuses on the stragglers root-causes that prevent us from achieving near ideal scaling in \package{MDAnalysis} beyond a single compute node.
In the present study, we are trying to find the root-cause of the straggler tasks and solutions through which we can improve performance and scaling.
  

%--------------------------------------------------

%Over the past few years a significant amount of research has been done trying to detect and mitigate stragglers \cite{Rosen2012, Dean2004, Chen2014, Bhandare2016, Kwon2012}. 
%There are also works to find the stragglers root-cause in many big data processing systems \cite{Ballani2011, Ananthanarayanan2014, Jeyakumar2013, Li2014, Zaharia2012}.

%Straggler root-cause has both internal and external factors. 
%Internal factors include heterogeneous capacity of worker nodes, and resource competition due to other tasks running on the same worker node. 
%External factors include resource competition due to co-hosted applications, input data skew, remote input or output source being too slow and faulty hardware \cite{Chen2014}.

%Spark introduces stragglers for a number of different reasons: 1) garbage collections~\cite{Kyong2017,Ousterhout2017}, 2) JVM positioning to cores~\cite{Kyong2017}, 
%2) the delay's introduced while the task moves from the scheduler to executing~\cite{Gittens2016}, 3) Disk I/O during shuffling, 4) Java's just-in-time compilation~\cite{Ousterhout2017}, 
%and 5) output skew \cite{Ousterhout2017}. 
%In addition to these reasons, stragglers on Spark have been attributed to the overall performance of the worker or competition between resources \cite{Yang2016}.

%Tuning resource allocation and tuning parallelism are other methods to improve Spark job performance. 
%However, manual tuning are laborious and imprecise. 
%Many frameworks suggest breaking the workload into many small tasks that are dynamically scheduled at runtime \cite{Rosen2012}. 
%This approach is only effective in systems with high-throughput, low-latency task schedulers and efficient data materialization though.
%In general, finer-grained splitting produces fewer stragglers; however, at some point the overhead of a large number of splits starts to dominate. 

%Some frameworks will identify the straggler nodes using a slow Node-Threshold. 
%They identify straggler node when its performance score is less than the average performance score of all nodes and therefore prevent to launch any tasks on these slow nodes. 
%The straggler node can be due to the faulty hardware or mis-configuration \cite{Dean2004}. 
%However, this solution only addresses hardware-based stragglers \cite{Chen2014}.

%For most of the factors causing stragglers, speculative execution and its improved versions are an effective way to solve the straggler problem. 
%Speculative execution (back-up tasks) is a replication-based reactive straggler mitigation technique that spawns redundant copies of the slow running tasks, hoping a copy will reach completion before the original. 
%This is the most prominently used in production clusters at Facebook and Microsoft Bing and is also implemented by both Hadoop and Google\textsc{\char13}s MapReduce \cite{Dean2004}.
%Google reported that speculative execution can improve job running times by $44\%$ \cite{Dean2004}.
%However, without any additional information, such reactive techniques can not differentiate between nodes that are inherently slow and nodes that are temporarily overloaded (i.e. input data skew or data imbalance) \cite{Chen2014, Bhandare2016}.

%Sampling or similar techniques can help estimate the data distribution to split the data more evenly. 
%However, such statistics are often expensive to collect, insufficiently precise, or obsolete. 
%Moreover, stragglers can happen even if the data distribution is balanced, due to unbalanced processing complexity for different parts of the data \cite{Khoshlessan:2017ab}.

%SkewTune is another technique for addressing data imbalance. 
%SkewTune identifies the task with the greatest expected remaining processing time when a node becomes idle. 
%The un-processed input data of this straggling task is then pro-actively repartitioned in a way that fully utilizes the nodes in the cluster and preserves 
%the ordering of the input data so that the original output can be reconstructed by concatenation \cite{Kwon2012}. 
%SkewTune can significantly reduce job runtime and adds little to no overhead in the absence of skew; however, the cost of fully reading the rest of a straggler's input can be prohibitive, in particular, making 
%the technique meaningless if the straggler was already dominated by the cost of reading the input.

%Garraghan et al. \cite{Garraghan2016} studied stragglers on Virtualized Cloud Data-centers. 
%Through statistical analysis they found that the most frequent reasons were high CPU utilization, disk utilization, unhandled I/O access requests and network package loss. 
%About $3\%$ to $6.5\%$ of the tasks were stragglers resulting to a $37.8\%$ to almost $50\%$ of jobs negatively impacted. 
%In this study, task stragglers are defined as tasks whose execution is $150\%$ of the median duration of all tasks within the same job and median task duration is used instead of mean task duration. 
%This is especially because median job execution duration is less affected by extreme execution times caused by stragglers.  

%The analysis performed on the trace data from Microsoft Bing's production cluster shows that $80\%$ of stragglers have a uniform probability of delay between $150-250\%$ 
%compared to the median task duration, with 10\% exhibiting a delay $1000\%$ greater than median task duration \cite{Ananthanarayanan2010}.

%Cloud data-flow deals with stragglers problem using dynamic work rebalancing. 
%However, this method has its own technical challenges which includes data consistency, intense systematic testing, as well as a careful design, and predicting how a task will progress over time \cite{Schmidt2016}.
%In addition, this approach is only effective in systems with high-throughput, low-latency task schedulers and efficient data materialization \cite{Rosen2012}.

%Blocked time analysis is used to measure how long each task spends blocked on a given resource. 
%These approach provides a per-task measurements and allows the understanding of straggler root-causes by correlating slow tasks with long blocked times \cite{Ousterhout2015}. 
%However, the work by \cite{Ousterhout2015} does not look at a vast range of workloads nor a wide range of cluster sizes.

%Intelligent scheduling \cite{AWE-WQ2014} is another approach to remove stragglers. Several components of task scheduling affect the scalability.
%The first is knowledge of a worker\textsc{\char13}s past task execution times. This allows to schedule tasks preferentially to the most performant machines. 
%The second involves task replication to remove the effect of straggling workers. 
%By monitoring the state of the workers either busy or idle ($w_{i}$) and the number of tasks waiting to be assigned a worker, task replication can be engaged when $w_{i}>0$.

%All of these previous studies are trying to mitigate stragglers through a wide variety of approaches.
%While a comprehensive solution will involve tricks to avoid causing stragglers. 
%This work focuses on the stragglers root-causes that prevent us from achieving near ideal scaling in \package{MDAnalysis} beyond a single compute node.
%In the present study, we are trying to find the root-cause of the straggler tasks and solutions through which we can improve performance and scaling.


