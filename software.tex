\label{Software}
\subsection{MPI for Python (\package{mpi4py})}
MPI for Python (\package{mpi4py}) is a Python wrapper written over Message Passing Interface (MPI) standard and allows any Python program to use multiple processors~\cite{Dalcin:2011aa, Dalcin:2005aa}.
Python is widely used in the scientific community because it facilitates rapid development of small scripts and code prototypes as well as development of large applications and highly portable and reusable modules and libraries.
Based on efficiency tests~\cite{Dalcin:2011aa, Dalcin:2005aa}, the performance degradation due to using \package{mpi4py} is not prohibitive and the overhead introduced by \package{mpi4py} is far smaller than the overhead associated to the use of interpreted versus compiled languages~\cite{GAiN}.
Overheads in \package{mpi4py} are small compared to C code if efficient raw memory buffers are used for communication~\cite{Dalcin:2011aa}.

\subsection{Application of Global Arrays}
The Global Arrays (GA) toolkit provides users with a language interface that allows them to distribute data while maintaining a type of global index space and programming syntax similar to what is available when programming on a single processor~\cite{GA}. The Global Arrays toolkit allows manipulation of distributed dense multi-dimensional arrays without explicitly defining communication and synchronization between processes.
Global Arrays in NumPy (GAiN)~\cite{GAiN} extends GA to Python through Numpy. 

The basic components of the Global Arrays toolkit are function calls to create (\texttt{ga\_create()}), copy data to (\texttt{ga\_put()}), from (\texttt{ga\_get()}), and between Global Arrays (\texttt{ga\_distribution()}, \texttt{ga\_scatter()}), and identify and access portions of Global Array data that are held locally (\texttt{ga\_access()}).
In addition, there are functions to destroy arrays (\texttt{ga\_destroy()}) and free up the memory originally allocated~\cite{GAiN}.

When a Global Array (GA) is created (\texttt{ga\_create()}) each process creates an array of the same shape and size, located in its memory space~\cite{GA}. 
The GA library keeps track of all these memory locations by recording a list of individual arrays, which are part of the GA. 
The user gets a pointer to the memory location of the GA by using \texttt{ga\_access()}.
By using this pointer ,it is possible to directly modify the data that is local to each process.
When a process tries to access a block of data, the request is first decomposed into individual blocks representing the contribution to the total request from the data held locally on each process (\textit{B. J. Palmer and J. Daily, personal communication}).
The requesting process then makes individual requests to each of the remote processes. 
The way data communication within the node and between the nodes happens depends on the configured runtime during installation.

Algorithm \ref{alg:GA} describes the RMSD algorithm in combination with a Global Arrays.
In this algorithm, we use Global Arrays instead of the message passing paradigm to investigate its effect to the communication cost. 

\begin{algorithm}[ht!]
	\scriptsize
	\caption{MPI-parallel Multi-frame RMSD using Global Arrays}
	\label{alg:GA}
	\hspace*{\algorithmicindent} \textbf{Input:}\emph{size}: Total number of frames assigned to each rank $N_{b}$\\
	\hspace*{\algorithmicindent} \emph{g\_a}: Initialized Global Arrays \\
	\hspace*{\algorithmicindent} \emph{xref0}: mobile group in the initial frame which will be considered as reference \\
	\hspace*{\algorithmicindent} \emph{start \& stop}: that tell which block of trajectory (frames) is assigned to each rank \\
	\hspace*{\algorithmicindent} \emph{topology \& trajectory}: files to read the data structure from \\
	\hspace*{\algorithmicindent}\textbf{Include:} \texttt{Block\_RMSD()} from Algorithm \ref{alg:RMSD}
	\begin{algorithmic}[1]
		
		\State bsize $\leftarrow$ ceil(trajectory.number\_frames / size)
		\State g\_a $\leftarrow$ ga.create(ga.C\_DBL, [bsize*size,2], "RMSD")
		\State buf $\leftarrow$ np.zeros([bsize*size,2], dtype=float)
		\State out $\leftarrow$ Block\_RMSD(topology, trajectory, xref0, start=start, stop=stop)
		\State ga.put(g\_a, out, (start,0), (stop,2))
		\If{rank == 0}
		\State buf $\leftarrow$ ga.get(g\_a, lo=None, hi=None)
		\EndIf
	\end{algorithmic}
\end{algorithm}

\subsection{MPI and Parallel HDF5}
HDF5 is a structured self-describing hierarchical data format which is the standard mechanism for storing large quantities of numerical data in Python ~\cite{pythonhdf5}.
Parallel HDF5 (\package{PHDF5}) typically sits on top of a MPI-IO layer and can use MPI-IO optimizations. 
In \package{PHDF5} all file accesses are coordinated through the MPI library; otherwise, multiple processes would compete over accessing over the same file on disk. 
MPI-based applications work by launching multiple parallel instances of the Python interpreter which communicate with each other via the MPI library. 
\package{HDF5} itself handles nearly all the details involved with coordinating file access when the shared file is opened through the \emph{mpio} driver.
In addition, MPI communicator should be supplied as well and the users also need to follow some constraints for data consistency \cite{pythonhdf5}.

MPI has two flavors of operation: collective, which means that all processes have to participate in the same order, and independent, which means each process can perform an operation or not and the order of execution does not matter~\cite{pythonhdf5}.
With \package{PHDF5}, modifications to file metadata must be done collectively and although all processes perform the same task, they do not wait until the others catch up~\cite{pythonhdf5}. 
Other tasks and any type of data operations can be performed independently by processes.
In the present study, we use independent operations.
