\label{system}
The test system contained the protein adenylate kinase with 214 amino acid residues and 3341 atoms in total \cite{Seyler:2014il}. 
The trajectory \cite{Seyler:2017aa} was in Gromacs XTC format trajectory (``600x'' in \citet{Khoshlessan:2017ab}) with a size of about 30 GB and 2,512,200
time frames (corresponding to $602.4 \sim \mu s$ simulated time) which represents a typical medium per-frame size but is very long for
current standards.

The experiments were executed on the XSEDE Supercomputers: \emph{SDSC Comet}, \emph{PSC Bridges} and \emph{SuperMIC}. 
SDSC Comet is a 2.7 PFlop/s cluster with 6,400 nodes in total.
The standard compute nodes consist of Intel Xeon E5-2680v3 processors, 128 GB DDR4 DRAM (64 GB per socket). 
The network topology is 56 Gbps FDR Infini-Band (IB).

PSC Bridges is a 1.35 PFlop/s cluster with four types of computational nodes.
Bridges computational nodes supply 1.3018 Pf/s and 274 TiB RAM.
The Regular Shared Memory (RSM) nodes consist of Intel Haswell (E5-2695 v3) processors, 128 GB DDR4 DRAM (64 GB per socket). 
The network topology is 12.37 Gbps Omni-Path Architecture (OPA).
\mknote{Giannis, Need some info on SuperMIC}
All the experiments are performed using compute nodes, RSM, \mknote{Giannis, what type of nodes did you use for the runs on SuperMIC?} on Comet, PSC Bridges and SuperMIC respectively in the present study.

\begin{table}
\centering
\begin{adjustbox}{max width=\textwidth}
\begin{tabular}{c c c c c c c}
  \toprule
            \bfseries\thead{Cluster} & \bfseries\thead{Nodes} & \bfseries\thead{Number} & \bfseries\thead{CPUs} &  \bfseries\thead{RAM} & \bfseries\thead{Network Topology}\\
  \midrule
    \bfseries SDSC Comet & Compute & 6400 & \makecell{2 Intel Xeon (E5-2680v3) CPUs \\ 12 cores/CPU, 2.5 GHz} &128 GB DDR4 DRAM & 56 Gbps IB \\
    \bfseries PSC Bridges & RSM & 752 & \makecell{2 Intel Haswell (E5-2695 v3) CPUs \\14 cores/CPU, 2.3 GHz} & 128 GB, DDR4-2133 & 12.37 Gbps OPA \\
    \bfseries SuperMIC & & & & \\
  \bottomrule
\end{tabular}
\end{adjustbox}
\caption[System configuration used in the present study]
{List of benchmarked clusters and their system configuration \mknote{Giannis, can you please add the sys config for SuperMIC}}
\label{tab:sys-config}
\end{table}


\subsection{Installing softwares on multi-user HPC systems}
\obnote{Include table with software and versions; add a paragraph in which you mention software, summarize how it was compiled, and reference the table.}
\mknote{Are the following enough?}
Different packages and libraries are used in the present study in order to achieve the desired performance. 
However, getting scientific softwares installed can be a very challenging task.
Lack of documentation and good software engineering practices, non-standard installation procedures, and lots of dependencies can contribute to the challenge of getting the software to work [???].
Scientists mostly care about the science and they are often not software engineers or system administrators. 
Therefore, the software they need should be easily accessible. 
This is not always the case, though.

In fact, many domain specific packages are not available through package manager in supercomputers and as a result,
we spent considerable amount of time getting packages dependencies to work in the process of our performance study. 
As a result, we provide detail information on how we managed to build these softwares.
This will let future works spend least amount of time for this purpose. 
Detailed information regarding the version of each library, its dependencies, the quality of its documentation, the time necessary for building and installing the packages are given in Table \ref{tab:version}. 

\begin{table}
\centering
\begin{adjustbox}{max width=\textwidth}
\begin{tabular}{c c c c c c c}
  \toprule
            \bfseries\thead{Package} & \bfseries\thead{Version} & \bfseries\thead{Description} & \bfseries\thead{Time to Result} & \bfseries\thead{Documentation} & \bfseries\thead{Installation} & \bfseries\thead{Dependencies}\\
  \midrule
   \bfseries GCC & 4.9.4 & GNU Compiler Collection & Fast & Excellent & \makecell{via configuration \\files, environment \\or command line options, \\ minimal configuration \\ is required} &--\\
   \bfseries OpenMPI & 1.10.7 & MPI Implementation & Fast & Excellent & \makecell{via configuration \\ files, environment \\or command line options, \\ minimal configuration \\ is required} &--\\
   \bfseries Global Array & 5.6.1 & Global Array & Slow & Good & \makecell{via configuration files, environment \\or command line options, \\ several optional configuration\\ settings available} &\makecell{MPI 1.x/2.x/3.x \\ implementation like \\ MPICH or Open MPI \\ built with shared/dynamic\\ libraries, GCC}\\
   \bfseries MPI4py & 3.0.0 & MPI for Python & Very Fast & Excellent & Conda Installation &\makecell{Python 2.7, \\or above, \\ MPI 1.x/2.x/3.x  \\ implementation like \\ MPICH or Open-MPI \\built with shared/dynamic \\libraries, Cython}\\
   \bfseries GA4py & 1.0 & Global Array for Python & Fast & Average & Python Setuptools &\makecell{Python 2.7, \\or above, \\ MPI 1.x/2.x/3.x  \\implementation like \\ MPICH or Open-MPI \\ built with shared/dynamic \\libraries, Cython,  \\MPI4py, Numpy} \\
   \bfseries PHDF5 & 1.10.1 & Parallel HDF5 & Slow & Excellent & \makecell{via configuration files, environment \\or command line options, \\ several optional configuration\\ settings available} &\makecell{MPI 1.x/2.x/3.x  \\ implementation like \\ MPICH or Open-MPI,  \\GNU, MPIF90,  \\MPICC, MPICXX}\\
   \bfseries H5py &  2.7.1 & Pythonic wrapper around the HDF5 & Very Fast & Excellent & Conda Installation & \makecell{Python 2.7, or above,\\ PHDF5, Cython}\\    
   \bfseries MDAnalysis & 0.17.0 & \makecell{Python library  \\ to analyze  \\trajectories from  \\MD simulations} & Very Fast & Excellent & Conda Installation & \makecell{Python $>=$2.7, or $<$3,\\ Cython, GNU, \\Numpy}\\
  \bottomrule
\end{tabular}
\end{adjustbox}
\caption[Version of the packages used in the present study]
{Detailed comparison on the dependency and installation of different packages used in the present study on a multi-user HPC system}
\label{tab:version}
\end{table}

From the libraries given in Table \ref{tab:version}, \emph{MPI4py}, \emph{H5py}, \emph{MDAnalysis} were the easiest to build. 
Users are able to get these packages installed through the \emph{Conda} package support.
\emph{OpenMPI}, \emph{GCC}, can be easily configured and installed.
The configure script supports a lot of different command line options, but the support for these software is very strong, they are widely used and the excellent documentation and discussion mailing lists provide users with great resources for consult, troubleshooting and tracking issues.
\emph{Global Array}, and \emph{PHDF5} have much slower installation especially because the softwares are built from source and variable configure options 
are required to support specific network interconnects and back-end run-time environments.
\emph{GA4py} has only one release, and does not provide users with strong documentation and there are still room for improvement for this package. 
 
Overall, the installation was successful on all clusters and we were able to observe similar performances (will be discussed in the result section) which shows the applicability of the softwares for achieving near ideal scaling.  

