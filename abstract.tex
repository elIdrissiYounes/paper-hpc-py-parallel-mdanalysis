\begin{abstract} 

In the biomolecular simulation community, analyzing trajectories from molecular dynamics (MD) simulations is becoming a bottleneck in the scientific workflow.
While the trajectory generation has seen large gains in performance by leveraging modern high performance computing (HPC) resources, acceleration of the analysis step has been lagging behind. 
Although many typical analysis tasks can be parallelized with a Map-Reduce approach we found that obtaining near linear strong scaling performance can be very difficult in practice because of \emph{straggler} processes whereby a few processes are much slower than the typical processes. 
Using the widely used \package{MDAnalysis} Python package as an example, we investigated a single program multiple data (SPMD) execution model where each process executes the same program to parallelize the Root Mean Square Distance (RMSD) algorithm using a Map-Reduce approach. 
We employed the Python language, which is widely used in the biomolecular simulation field, and focus on MPI-based implementations.  
Stragglers were less prevalent for compute-bound workloads (as measured by the ratio of compute to I/O time and the ratio of compute to communication time).
We found that stragglers were primarily caused by either excessive MPI communication costs or excessive time to open the single shared trajectory file whereas both the computation and the ingestion of data exhibited close to ideal strong scaling behavior.
We improved performance by (1) reducing the communication cost with the \package{Global Arrays} (GA) toolkit and (2) testing two different approaches to improve file access.
The first approach is splitting the trajectory into as many trajectory segments as number of processes (``subfiling'').
The second approach is through MPI-based approach using Parallel HDF5 where we examine the performance through independent I/O.
Applying these strategies, we obtained near ideal scaling and performance up to 384 cores.
We provide insights, guidelines, and strategies to the biomolecular simulation community on how to take advantage of the available HPC resources to gain good performance.
\end{abstract}

