\label{sec:introduction}
The increase in computational power coupled with sophisticated algorithms has lead rapid increase in the amount of data produced by MD simulations. 
Typical trajectory sizes from MD simulations range from gigabytes to terabytes. 
Therefore, analyzing these trajectories has become a very tedious process in many workflows and as a result people are trying to look for state of the art HPC tools (MPI and OpenMP) or Big Data ecosystem to tackle this problem.
The need for parallel programming and running these programs on parallel architectures is obvious, however, efficiently programming for a parallel environment can be a very daunting task. 

\package{MDAnalysis} \cite{Gowers:2016aa,Michaud-Agrawal:2011fu} is a widely used open-source Python library to analyze molecular dynamics (MD) simulations. 
\package{MDAnalysis} allows analysis of different file formats for trajectories generated by various packages for molecular dynamic simulations. 

In our previous study, we used a parallel map-reduce approach to study the performance of RMSD task \cite{Khoshlessan:2017ab}. 
We previously looked at the \package{Dask} library \cite{Rocklin:2015aa}, which splits a computation in tasks and generates directed acyclic graphs (DAG) of these tasks that can be executed on a range of schedulers. 
We also implemented the parallel analysis scheme with MPI, using the \package{mpi4py} package \cite{Dalcin:2011aa, Dalcin:2005aa}. 
For both Dask and MPI we found that our benchmark task, the calculation of the minimum C$_{\alpha}$ RMSD for a
subset of the residues in the enzyme adenylate kinase from a long MD simulation, only showed good strong scaling within a single node (up
to 24 cores on \emph{SDSC Comet}).
However, with a single compute node we are limited by the resources for executing a given problem.
Distributed computing, allows parallelizing our problems for larger problem sizes and lead to performance gains.
But, as soon as we extend the computation beyond a single node, performance drops due to \emph{stragglers} tasks, a subset of Dask worker processes or MPI ranks that are significantly slower than the mean execution time of all tasks, increasing the total time to solution.
Stragglers significantly impede job completion time and are a big challenge toward achieving improved performance.

MPI should have, in principle, close to ideal scaling for a pleasingly parallel task such as the analysis of trajectory blocks, and does not require additional considerations of, e.g., scheduler performance as for Dask. 
Therefore, in the present study, we analyze the MPI case in more detail to better understand the origin of the stragglers.

We want to provide simple and robust parallelization approaches to analyzing molecular dynamics (MD) trajectories, in order to remove a narrowing bottle neck in the bio-molecular simulation field. 
We have selected two of the most common map-reduce algorithms in \package{MDAnalysis} one of which is I/O bound and the other is compute bound.
We use SPMD paradigm to parallelize theses two algorithms on HPC resources.
With SPMD, each process executes essentially the same code but on a different part of the data. 
We use Python, a machine-independent, byte-code interpreted, object-oriented programming (OOP) language, which is well-established in HPC parallel environments [GA thesis]. 
Based on our initial analysis there is an important performance parameter,  $t_{Compute}$/$t_{IO}$, that determines whether we observe stragglers.
We show this behavior using RMSD and dihedral featurization algorithms.
If $t_{Compute}$/$t_{IO}$  $\gg 1$, the algorithm scales very well, otherwise it does not scale beyond one node. 
For the algorithms with small $t_{Compute}$/$t_{IO}$, we need to come up with strategies to improve scaling and overcome straggler problems.
Looking at the timing distribution across all ranks we noticed that communication and I/O are the two main scalability bottlenecks.
Taking advantage of Global Array toolkit, each rank places its data in shared memory which allows direct access using the global address space rather than through a message-passing protocol. 
This reduces communication cost noticeably.
Although Global Array toolkit is very helpful for improving the performance it is still not enough because I/O remains a bottleneck.
Our data shows that I/O time does not scale beyond one node. 
Due to the large file size and memory limit, processes are not able to load the whole trajectory into memory at once, and as a result each process is only allowed to load one frame into memory at a time.
Therefore, with large number of frames, there will be a lot of file access requests and when the compute time is small with respect to I/O, then I/O can be a major issue.
Hence, we needed to find ways to improve I/O scaling beyond a single compute node.
In order to improve I/O scaling, we came up with two approaches: MPI-based approach using Parallel HDF5, and splitting our trajectory to as many trajectory segments as the number of processes. 
We provide the detail on these approaches on the following sections.
But, both approaches significantly improved the performance and we were able to achieve near ideal scaling.