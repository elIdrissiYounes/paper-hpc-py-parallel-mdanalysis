\label{sec:introduction}
<<<<<<< HEAD
Molecular dynamics simulations are the most powerful and fundamental tool among a wide variety of techniques that enable scientists to achieve detailed info regarding the function of bio-molecules in living cells [QwickMD].
MD is widely used to understand structure-to-function relationships of complex protein systems.
Data analysis tools and libraries have been developed to extract the desired information from these MD simulations ~\cite{Gowers:2016aa,Michaud-Agrawal:2011fu,cpptraj-2013,himach-2008} [VMD, MDtraj].
\package{MDAnalysis} \citep{Gowers:2016aa,Michaud-Agrawal:2011fu} is an open-source object-oriented Python library for structural and temporal analysis of molecular dynamics simulation trajectories and individual protein structures. 

Development of powerful parallel supercomputers has significantly increased the physical and time scales of MD simulations.
Present simulation times are close to biologically relevant ones and this has lead to rapid increase in the amount of data produced by MD simulations. 
As a result, typical trajectory sizes from MD simulations range from Gigabytes to Terabytes~\cite{Cheatham:2015}. 
Therefore, parallel algorithms and frameworks are the key to meeting the scalability and performance requirements for analyzing these trajectory files. 
=======
The increase in computational power coupled with sophisticated algorithms has lead rapid increase in the amount of data produced by MD simulations. 
Typical trajectory sizes from MD simulations range from Gigabytes to Terabytes~\cite{Cheatham:2015}. 
Therefore post-processing and analyzing these trajectories in order to obtain metastable states, dynamical modes, transition pathways, structure-function relationships of biomolecules and other key information that can be extracted from MD simulations has become a very tedious process in many workflows and requires great amount of time when performed serially. 
As a result people are trying to look for state of the art HPC tools (MPI and OpenMP) or Big Data ecosystem to speed up post processing task of MD trajectories.
\gpnote{It is difficult for me to parse this sentence and understand the point. I do not understand how from introducing the problem size, it is concluded that parallelizing becomes important and why programming efficiently can be a very daunting task. I feel like something is missing.} \mknote{tried to address it. Does that make sense now?}\gpnote{Better definitely, although I suggest challenging compared to daunting}
However, efficiently programming for a parallel environment and achieving near ideal scaling performance can be a very daunting task. 
>>>>>>> 2fa730f5f3d2510ea958bb169176fbf54b0c9792

\package{MDAnalysis} does not provide parallel analysis of these trajectories.
Thus, post-processing and analyzing these trajectories can become a very tedious process and requires great amount of time when performed serially. 
As a result, we are trying to look for state of the art HPC tools (MPI and OpenMP) or Big Data ecosystem to speed up post processing task of MD trajectories.
However, efficiently programming for a parallel environment and achieving near ideal scaling performance can be a very daunting task. 

In our previous study, we used a parallel map-reduce approach to study the performance of RMSD task \cite{Khoshlessan:2017ab, ICCP-2018}. 
We previously looked at the \package{Dask} library \cite{Rocklin:2015aa}, which splits a computation in tasks and generates directed acyclic graphs (DAG) of these tasks that can be executed on a range of schedulers. 
We also implemented the parallel analysis scheme with MPI, using the \package{mpi4py} package \cite{Dalcin:2011aa, Dalcin:2005aa}. 
For both Dask and MPI we found that our benchmark task, the calculation of the minimum C$_{\alpha}$ RMSD for a
subset of the residues in the enzyme adenylate kinase from a long MD simulation, only showed good strong scaling within a single node (up to 24 cores on \emph{SDSC Comet}).
However, with a single compute node we are limited by the resources for executing a given problem.
Distributed computing, allows parallelizing our problems for larger problem sizes and lead to performance gains.
But, as soon as we extend the computation beyond a single node, performance drops due to \emph{stragglers} tasks, a subset of worker processes\gpnote{Is it really a problem of Dask, or how it was used? This points to Dask. I believe it was more how IO was happening, which points to how it was used.}\mknote{corrected}\gpnote{You still mention worker processes. I suggest to remove it. In MPI there are no worker processes by default} that are significantly slower than the mean execution time of all tasks, increasing the total time to solution.
Stragglers significantly impede job completion time and are a big challenge toward achieving improved performance.

\gpnote{This hides the assumption that MPI is used fully, i.e. using all of its features. I do not think it helps.}\mknote{I do not understand what do you mean?}\gpnote{Were all MPI capabilities used fully in the implementations, i.e. MPI-IO, asynchronous communications, inter and intra node communication? Saying that MPI is the issue with out informing the reader what you used in the initial implementations lead to the wrong assumptions} MPI should have, in principle, close to ideal scaling for a pleasingly parallel task such as the analysis of trajectory blocks, and does not require additional considerations of, e.g., scheduler performance as for Dask. 
Therefore, in the present study, we analyze the MPI case in more detail to better understand the origin of the stragglers.
<<<<<<< HEAD
We want to provide simple and robust parallelization approaches to analyzing molecular dynamics (MD) trajectories, in order to speed-up post-processing of MD trajectories in \package{MDAnalysis} library. 
=======

We want to provide simple and robust parallelization approaches to analyzing molecular dynamics (MD) trajectories, in order to speed-up post-processing of MD trajectories.
>>>>>>> 2fa730f5f3d2510ea958bb169176fbf54b0c9792

We have selected two of the algorithms in \package{MDAnalysis} one of which is I/O bound (RMSD) and the other is compute bound (Dihedral Featurization).
We use SPMD paradigm to parallelize theses two algorithms on HPC resources.
With SPMD, each process executes essentially the same code but on a different part of the data. 
We use Python, a machine-independent, byte-code interpreted, object-oriented programming (OOP) language, which is well-established in HPC parallel environments \cite{GAiN}. 
Based on our initial analysis, there are two important performance parameters,  $t_{Compute}$/$t_{IO}$ and $t_{Compute}$/$t_{Communication}$ which are the ratio of computational to I/O load, as measured by the time spent on the computation versus the time spent on reading data from the file system, that determines whether we observe stragglers and the ratio of computational to communication load respectively.
We show this behavior using RMSD and dihedral featurization algorithms.
If $t_{Compute}$/$t_{IO}$  $\gg 1$, the algorithm scales very well, otherwise it does not scale beyond a single node. 
For the algorithms with small $t_{Compute}$/$t_{IO}$, we need to come up with strategies to improve scaling and overcome straggler problems.
Looking at the timing distribution across all ranks we noticed that communication and I/O are the two main scalability bottlenecks.

Taking advantage of Global Array toolkit we were able to reduce communication cost noticeably.
In addition, our data show that I/O time does not scale beyond one node. 
\gpnote{I consider MPI-style IO as the standard and CPPTraj in the realted work shows that. I suggest you either clarify it a bit or say that you used them.}\mknote{Can you clarify more? What do you exactly mean here?}
\gpnote{It is simple. Writing an MPI program that requires processes to read and write to the same file, it is required to use MPI-IO. If not write and read from different files. Also, in the related work you refer to CPPTraj, which already uses MPI-IO and HDF5 to read and write trajectories. As a result the approaches used are not necessarily something new. It might be better to rephrase the next sentence a bit.}
In order to improve I/O scaling, we came up with two approaches: MPI-based approach using Parallel HDF5 \cite{pythonhdf5}, and splitting our trajectory to as many trajectory segments as the number of processes. 
We provide the detail on these approaches on the following sections.
But, both approaches significantly improved the performance and we were able to achieve near ideal scaling.