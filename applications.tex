\label{use_cases}

\subsection{MDAnalysis}
Simulation data exist in trajectories in the form of three dimensional time series (atoms positions and velocities), and these come in a plethora of different and idiosyncratic file formats. 
\package{MDAnalysis} is a widely used open source Python library to analyze these trajectory files with an object oriented interface. 
The package is written in Python (compatible with version 2.7 and 3.4+), with time critical code in C/Cython. 
\label{sec:mda}

\subsubsection{Root Mean Square Distance (RMSD)}
The calculation of the root mean square distance (\textbf{RMSD}) for C$_{\alpha}$ atoms after
optimal superposition with the QCPROT algorithm \cite{Liu:2010kx,Theobald:2005vn} is commonly required in the analysis of molecular dynamics simulations (Algorithm \ref{alg:RMSD}). 
The task used for the purpose of our benchmark is the MDAnalysis.analysis.rms.rmsd() function from the MDAnalysis.analysis.rms module (implemented in Cython \cite{Gowers:2016aa}).  
This function computes the RMSD between two sets of coordinates using the fast QCPROT algorithm to optimally superimpose two structures and then calculates the RMSD. 

To this, the protein structure (selected C$_{\alpha}$ atoms) in the initial frame will be considered as the reference and as the mobile group at other time steps. 
The superposition is done in the following way:
First, the mobile group is translated so that its center of mass coincides with the one of reference. 
Second, a rotation matrix is computed that spatially aligns the mobile group to reference which minimizes the RMSD between the coordinates of the mobile group and reference structure. 
Finally, all atoms in mobile group are shifted and rotated.
For each frame, a non-negative floating point number is calculated and the final result is a time-series of the RMSD. 
RMSD values show how rigid the domains in a protein structure are, during the transition. 
The order of complexity for RMSD algorithm \ref{alg:RMSD} is $T \times N^{2}$ \cite{rmsd} where T is the number of frames in the trajectory and N the number of particles in a 
frame.

\begin{algorithm}[ht]
	\scriptsize
	\caption{MPI-parallel Multiframe RMSD Algorithm\gpnote{This is my suggestion. I believe some of the comments are very naive for the BioSim community. Please remove them}}
	\label{alg:RMSD_sug}
	\hspace*{\algorithmicindent} \textbf{Input:} \emph{size}: Total number of frames \\
	\hspace*{\algorithmicindent} \emph{ref}: mobile group in the initial frame which will be considered as reference \\
	\hspace*{\algorithmicindent} \emph{start \& stop}: Starting and stopping frame index\\
	\hspace*{\algorithmicindent} \emph{topology \& trajectory}: files to read the data structure from \\
	\hspace*{\algorithmicindent} \textbf{Output:} Calculated RMSD arrays
	\begin{algorithmic}[1]
		\Procedure{$Block\_RMSD$}{topology, trajectory, $ref$, start, stop}                       
		\State u = Universe(topology, trajectory)\Comment{u hold all the information of the physical system}
		\State $g$ = u.frames[start:stop]\Comment{The selected frames}
		\For{$\forall iframe$ in $g$}
		\State $results[iframe] = RMSD(g, ref)$
		\EndFor
		\State \Return results
		\EndProcedure
		\\        
		\State MPI Init
		\State rank = MPI\_WORLD.rank
		\State index = indices of mobile atom group
		\State xref0 = mobile.positions-mobile.center\_of\_mass()
		\State out = Block\_RMSD(topology, trajectory, xref0, start=start, stop=stop)
		\\
		\State Gather(out, RMSD\_data, root=0)\Comment{Gather all data to rank 0}
		\State MPI Finalize
	\end{algorithmic}
\end{algorithm}

\begin{algorithm}[ht]
	\scriptsize
    \caption{RMSD Algorithm\gpnote{I agree with Oliver. This is a bit difficult to read. I would suggest to take out the Python specifics like MDAnalysis.analysis.rms.rmsd. I can fix it if you want.}}
    \label{alg:RMSD}
    \hspace*{\algorithmicindent} \textbf{Input:} \emph{mobile}: the desired atom groups to perform RMSD on them \\ 
    \hspace*{\algorithmicindent} \emph{bsize}: Total number of frames assigned to each rank \\
    \hspace*{\algorithmicindent} \emph{xref0}: mobile group in the initial frame which will be considered as rerference \\
    \hspace*{\algorithmicindent} \emph{start \& stop}: that tell which block of trajectory (frames) is assigned to each rank \\
    \hspace*{\algorithmicindent} \emph{topology \& trajectory}: files to read the data structure from \\
    \hspace*{\algorithmicindent} \textbf{Output:} Calculated RMSD arrays
    \begin{algorithmic}[1]
        \Procedure{$Block\_RMSD$}{index, topology, trajectory, $xref0$, start=None, stop=None, step=None}                       
        \State \texttt{start00 = time.time()}
        \State \texttt{u = Universe(topology, trajectory)}
        \State \texttt{g = u.atoms[index]}
        \\
        \State \texttt{start1 = time.time()}
        \State \texttt{start0 = start1}
        \For{$\forall iframe, ts$ in $enumerate(u.trajectory[start:stop:step])$}
        \State \texttt{start2 = time.time()}
        \State \texttt{results[iframe, :] = ts.time, MDAnalysis.analysis.rms.rmsd(g.positions, xref0, center=True, superposition=True)}
        \State \texttt{t\_comp[iframe] = time.time()-start2}\Comment{Compute per frame}
        \State \texttt{t\_IO[iframe] = start2-start1}\Comment{I/O per frame}
        \State \texttt{start1 = time.time()}
        \EndFor
        
        \State \texttt{start3 = time.time()}
        \State \Return \texttt{results}
        \EndProcedure
        \\        
        \State \texttt{start4 = time.time()}
        \State \texttt{mobile =  Select C$_{\alpha}$ atoms}
        \State \texttt{index = indices of mobile atom group}
        \State \texttt{xref0 = mobile.positions-mobile.center\_of\_mass()}
	\State \texttt{out = Block\_RMSD(index, topology, trajectory, xref0, start=start, stop=stop, step=1)}
	\\
	\State \texttt{start5 = time.time()}
	\If{rank == 0}
   	     \State \texttt{data1 = np.zeros([size*bsize,2], dtype=float)}
	\Else
   	     \State \texttt{data1 = None}
              \State \texttt{comm.Gather(out[0], data1, root=0)}
        \EndIf
         \\
         \State \texttt{start6 = time.time()}
    \end{algorithmic}
\end{algorithm}


\subsubsection{Dihedral Featurization}
As a real-world compute-bound task we investigated \textbf{Dihedral featurization} \cite{Sittel:2014aa} (Algorithm \ref{alg:Dihedral}) whereby a time series of
feature vectors consisting of the two backbone dihedral angles per residue ($\phi_{i}$ and $\psi_{i}$) is calculated for all 212
non-terminal residues. For each frame, an array of dihedral angles is calculated where for later convenience, an angle $\theta_{i}$ is
actually represented as $(\cos\theta_{i}, \sin\theta_{i})$. 
The order of complexity for Dihedral featurization algorithm (Algorithm \ref{alg:Dihedral}) is $T \times N$. 

\begin{algorithm}[ht!]
	\scriptsize
	\caption{Dihedral Featurization\gpnote{This is my suggestion}}
	\label{alg:Dihedral}
	\hspace*{\algorithmicindent} \textbf{Input:} \emph{mobile}: the desired atom groups to perform RMSD on them \\ 
	\hspace*{\algorithmicindent} \emph{start \& stop}: that tell which block of trajectory (frames) is assigned to each rank \\
	\hspace*{\algorithmicindent} \emph{topology \& trajectory}: files to read the data structure from \\
	\hspace*{\algorithmicindent} \textbf{Output:} Calculated Dihedral Angles
	\begin{algorithmic}[1]
		\Procedure{$angle2sincos$}{x}     
		\State \Return $\cos{x},\: \sin{x}$
		\EndProcedure
		\\
		\Procedure{$residues\_to\_dihedrals$}{residues}\gpnote{This is not called anywhere. It should either be removed or connected to the rest of the listing}
		\State List angles
		\For{$\forall res$ in $residues$}
		\State Append $(\phi (res),\: \psi (res))$ in angles
		\EndFor
		\State \Return angles
		\EndProcedure
		\\
		\Procedure{$featurize\_dihedrals$}{dihedrals}     
		\State \texttt{angles = [dihedral.value() for dihedral in dihedrals]}
		\State \Return \texttt{angle2sincos(angles)}
		\EndProcedure
		\\
		\Procedure{$Block\_Dihedral$}{topology, trajectory, $ref$, start, stop}                       
		\State u = Universe(topology, trajectory)
		\State $g$ = u.frames[start:stop]\Comment{The selected frames} 
		\State List $results$
		\For{$\forall frame$ in $g$}
		\State $D_{angles}=featurize\_dihedrals(dihedrals)$
		\State Append $D_{angles}$ in $results$ 
		\EndFor
		\EndProcedure
		\\      
		\State MPI Init 
		\State rank = MPI\_WORLD.rank
		\State index = indices of mobile atom group
		\State xref0 = mobile.positions-mobile.center\_of\_mass()
		\State out = Block\_Dihedral(topology, trajectory, xref0, start=start, stop=stop)
		\\
		\State Gather(out, RMSD\_data, root=0)\Comment{Gather all data to rank 0}
		\State MPI Finalize
	\end{algorithmic}
\end{algorithm}

\begin{algorithm}[ht!]
	\scriptsize
    \caption{Dihedral Featurization}
    \label{alg:Dihedral}
    \hspace*{\algorithmicindent} \textbf{Input:} \emph{mobile}: the desired atom groups to perform RMSD on them \\ 
    \hspace*{\algorithmicindent} \emph{start \& stop}: that tell which block of trajectory (frames) is assigned to each rank \\
    \hspace*{\algorithmicindent} \emph{topology \& trajectory}: files to read the data structure from \\
    \hspace*{\algorithmicindent} \textbf{Output:} Calculated Dihedral Angles
    \begin{algorithmic}[1]
        \Procedure{$angle2sincos$}{x}     
        \State \Return \texttt{[np.cos(x), np.sin(x)]}
        \EndProcedure
        \\
        \Procedure{$residues\_to\_dihedrals$}{residues}     
        \For{$\forall res$ in $residues$}
        \State \texttt{list($\phi (res)$, $\psi (res)$)}
        \EndFor
        \EndProcedure
        \\
        \Procedure{$featurize\_dihedrals$}{dihedrals}     
        \State \texttt{angles = [dihedral.value() for dihedral in dihedrals]}
        \State \Return \texttt{angle2sincos(angles)}
        \EndProcedure
        \\
        \Procedure{$Block\_Dihedral$}{index, topology, trajectory, $xref0$, start=None, stop=None, step=None}                       
        \State \texttt{start00 = time.time()}
        \State \texttt{clone = mda.Universe(topology, trajectory)}
        \State \texttt{g = clone.atoms[index]}
        \\
        \State \texttt{start1 = time.time()}
        \State \texttt{start0 = start1}
        \For{$\forall iframe, ts$ in $enumerate(clone.trajectory[start:stop:step])$}
        \State \texttt{start2 = time.time()}
        \State \texttt{Append in $results$ $featurize\_dihedrals(dihedrals)$}
        \State \texttt{t\_comp[iframe] = time.time()-start2}\Comment{Compute per frame}
        \State \texttt{t\_IO[iframe] = start2-start1}\Comment{I/O per frame}
        \State \texttt{start1 = time.time()}
        \EndFor
        
        \State \texttt{start3 = time.time()}
        \EndProcedure
        \\        
        \State \texttt{start4 = time.time()}
	\State \texttt{out = Block\_Dihedral(index, topology, trajectory, xref0, start=start, stop=stop, step=1)}
	\\
	\State \texttt{start5 = time.time()}
	\If{rank == 0}
   	     \State \texttt{data1 = np.zeros([size*bsize,np.shape(out)[1]], dtype=float)}
	\Else
   	     \State \texttt{data1 = None}
             \State \texttt{comm.Gather(out[0], data1, root=0)}
        \EndIf
         \\
         \State \texttt{start6 = time.time()}
    \end{algorithmic}
\end{algorithm}