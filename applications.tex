\label{use_cases}

\subsection{MDAnalysis}
\label{sec:mda}

Simulation data exist in trajectories in the form of time series of atoms positions and sometimes velocities, and these come in a plethora of different and idiosyncratic file formats. 
\package{MDAnalysis} \citep{Gowers:2016aa,Michaud-Agrawal:2011fu} is a widely used open source Python library to analyze these trajectory files with an object oriented interface. 
The package is written in Python, with time critical code in C/Cython. 
\package{MDAnalysis} supports most file formats of simulation packages including CHARMM, Gromacs, Amber, and NAMD and the Protein Data Bank format and enables accessing data stored in trajectories. 
 
\subsubsection{Root Mean Square Distance (RMSD)}
The calculation of the root mean square distance (\textbf{RMSD}) for C$_{\alpha}$ atoms for a subset of the residues after 
optimal superposition with the QCPROT algorithm \cite{Liu:2010kx,Theobald:2005vn} (implemented in Cython \cite{Gowers:2016aa}) is commonly required in the analysis of molecular dynamics simulations (Algorithm \ref{alg:RMSD}). 
The task used for the purpose of our benchmark is the $\text{RMSD}=\sqrt{\frac{1}{N}\sum_{i=1}^{N}\delta_{i}^{2}}$ implemented in MDAnalysis.analysis.rms module where $\delta_{i}$ is the distance between atom $i$ and a reference structure.  

To this, the protein structure (selected C$_{\alpha}$ atoms) in the initial frame will be considered as the reference and as the mobile group at other time steps. 
In the RMSD algorithm, a translation and rigid body rotation are found that minimize the RMSD between the mobile coordinates and the reference coordinates. 
The RMSD of this optimum superposition is reported. 
The process is repeated for each trajectory frame, with the output being a time series of RMSD values.
The RMSD is used to show the rigidity of protein domains and more generally characterizes structural changes.
The order of complexity for RMSD algorithm \ref{alg:RMSD} is $\mathcal{O}(T \times N^{2})$ \cite{Liu:2010kx} where $T$ is the number of frames in the trajectory and $N$ the number of particles in a frame.

\begin{algorithm}[ht!]
	\scriptsize
	\caption{MPI-parallel Multi-frame RMSD Algorithm}
	\label{alg:RMSD}
	\hspace*{\algorithmicindent} \textbf{Input:} \emph{size}: Total number of frames \\
	\hspace*{\algorithmicindent} \emph{ref}: mobile group in the initial frame which will be considered as reference \\
	\hspace*{\algorithmicindent} \emph{start \& stop}: Starting and stopping frame index\\
	\hspace*{\algorithmicindent} \emph{topology \& trajectory}: files to read the data structure from \\
	\hspace*{\algorithmicindent} \textbf{Output:} Calculated RMSD arrays
	\begin{algorithmic}[1]
		\Procedure{$Block\_RMSD$}{topology, trajectory, $ref$, start, stop}                       
		\State u $\leftarrow$ Universe(topology, trajectory)\Comment{u hold all the information of the physical system}
		\State $g$ $\leftarrow$ u.frames[start:stop]
		\For{$\forall iframe$ in $g$}
		\State $results[iframe] \leftarrow RMSD(g, ref)$
		\EndFor
		\State \Return results
		\EndProcedure
		\\        
		\State MPI Init
		\State rank $\leftarrow$ rank ID
		\State index $\leftarrow$ indices of mobile atom group
		\State xref0 $\leftarrow$ Reference atom group\textsc{\char13}s position
		\State out $\leftarrow$ Block\_RMSD(topology, trajectory, xref0, start=start, stop=stop)
		\\
		\State Gather(out, RMSD\_data, rank\_ID=0)
		\State MPI Finalize
	\end{algorithmic}
\end{algorithm}

